\documentclass[tikz,border=3mm]{article}
\usepackage{tikz}
\usepackage{tikz-3dplot}
\usetikzlibrary{fpu}
\usepackage{tikz-3dplot-circleofsphere}

\makeatletter

\usepackage{xparse}

\def\RadToDeg{{180/3.14159265358979323846264338327950288}}
\def\DegToRad{{3.14159265358979323846264338327950288/180}}
%\newcommand{\qlscXyToRaDec}[3] {% face,x,y
%	\def
%}

% ref: https://tex.stackexchange.com/a/279500/162934
\newcommand{\sphToCart}[3]
        {
          \def\rpar{#1}
          \def\thetapar{#2}
          %\def\phipar{#3}
          \def\phipar{90+(-#3)} % convert angle to declination

          \pgfmathsetmacro{\x}{\rpar*sin(\phipar)*cos(\thetapar)}
          \pgfmathsetmacro{\y}{\rpar*sin(\phipar)*sin(\thetapar)}
          \pgfmathsetmacro{\z}{\rpar*cos(\phipar)}
        }
        
\pgfmathdeclarefunction{isfore}{3}{%
\begingroup%
\pgfkeys{/pgf/fpu,/pgf/fpu/output format=fixed}%
\pgfmathparse{%
sign(((\the\pgf@yx)*(\the\pgf@zy)-(\the\pgf@yy)*(\the\pgf@zx))*(#1)+
((\the\pgf@zx)*(\the\pgf@xy)-(\the\pgf@xx)*(\the\pgf@zy))*(#2)+
((\the\pgf@xx)*(\the\pgf@yy)-(\the\pgf@yx)*(\the\pgf@xy))*(#3))}%
\pgfmathsmuggle\pgfmathresult\endgroup%
}%

\tikzset{great circle arc/.cd,
    theta1/.initial=0,
    phi1/.initial=0,
    theta2/.initial=0,
    phi2/.initial=30,
    r/.initial=R,
    fore/.style={draw=white,semithick},
    back/.style={draw=gray,thick,dashed,opacity=0.0}}

\newcommand\GreatCircleArc[2][]{%
\tikzset{great circle arc/.cd,#2}%
\def\pv##1{\pgfkeysvalueof{/tikz/great circle arc/##1}}%
 % Cartesian coordinates of the first point (A) 
\pgfmathsetmacro\tikz@td@Ax{\pv{r}*cos(\pv{theta1})*cos(\pv{phi1})}%
\pgfmathsetmacro\tikz@td@Ay{\pv{r}*cos(\pv{theta1})*sin(\pv{phi1})}%
\pgfmathsetmacro\tikz@td@Az{\pv{r}*sin(\pv{theta1})}%
 % Cartesian coordinates of the second point (B) 
\pgfmathsetmacro\tikz@td@Bx{\pv{r}*cos(\pv{theta2})*cos(\pv{phi2})}%
\pgfmathsetmacro\tikz@td@By{\pv{r}*cos(\pv{theta2})*sin(\pv{phi2})}%
\pgfmathsetmacro\tikz@td@Bz{\pv{r}*sin(\pv{theta2})}%
 % cross product C=AxB 
\pgfmathsetmacro\tikz@td@Cx{(\tikz@td@Ay)*(\tikz@td@Bz)-(\tikz@td@By)*(\tikz@td@Az)}%
\pgfmathsetmacro\tikz@td@Cy{(\tikz@td@Az)*(\tikz@td@Bx)-(\tikz@td@Bz)*(\tikz@td@Ax)}%
\pgfmathsetmacro\tikz@td@Cz{(\tikz@td@Ax)*(\tikz@td@By)-(\tikz@td@Bx)*(\tikz@td@Ay)}%
 % normalize C to have length r
\pgfmathsetmacro\pgfutil@tempa{sqrt((\tikz@td@Cx)*(\tikz@td@Cx)+(\tikz@td@Cy)*(\tikz@td@Cy)+(\tikz@td@Cz)*(\tikz@td@Cz))/\pv{r}}%
\pgfmathsetmacro\tikz@td@Cx{\tikz@td@Cx/\pgfutil@tempa}%
\pgfmathsetmacro\tikz@td@Cy{\tikz@td@Cy/\pgfutil@tempa}%
\pgfmathsetmacro\tikz@td@Cz{\tikz@td@Cz/\pgfutil@tempa}%
 % angle between A and B
\pgfmathsetmacro\tikz@td@AdotB{((\tikz@td@Ax)*(\tikz@td@Bx)+
    (\tikz@td@Ay)*(\tikz@td@By)+(\tikz@td@Az)*(\tikz@td@Bz))/(\pv{r}*\pv{r})}% 
\pgfmathsetmacro\tikz@td@angle{acos(\tikz@td@AdotB)}%   
 % cross product D=AxC
\pgfmathsetmacro\tikz@td@Dx{(\tikz@td@Ay)*(\tikz@td@Cz)-(\tikz@td@Cy)*(\tikz@td@Az)}%
\pgfmathsetmacro\tikz@td@Dy{(\tikz@td@Az)*(\tikz@td@Cx)-(\tikz@td@Cz)*(\tikz@td@Ax)}%
\pgfmathsetmacro\tikz@td@Dz{(\tikz@td@Ax)*(\tikz@td@Cy)-(\tikz@td@Cx)*(\tikz@td@Ay)}%
\pgfmathsetmacro\pgfutil@tempa{sqrt((\tikz@td@Dx)*(\tikz@td@Dx)+(\tikz@td@Dy)*(\tikz@td@Dy)+(\tikz@td@Dz)*(\tikz@td@Dz))/\pv{r}}%
\pgfmathsetmacro\tikz@td@Dx{\tikz@td@Dx/\pgfutil@tempa}%
\pgfmathsetmacro\tikz@td@Dy{\tikz@td@Dy/\pgfutil@tempa}%
\pgfmathsetmacro\tikz@td@Dz{\tikz@td@Dz/\pgfutil@tempa}%
 %\typeout{A=(\tikz@td@Ax,\tikz@td@Ay,\tikz@td@Az),B=(\tikz@td@Bx,\tikz@td@By,\tikz@td@Bz),C=(\tikz@td@Cx,\tikz@td@Cy,\tikz@td@Cz)}
 %\typeout{\tikz@td@AdotB,\tikz@td@angle}
\edef\pgfutil@tempa{0}%
\pgfmathtruncatemacro{\pgfutil@tempd}{isfore(\tikz@td@Ax,\tikz@td@Ay,\tikz@td@Az)}%
\ifnum\pgfutil@tempd=-1\relax
\edef\tikz@td@lsthidcoords{(\tikz@td@Ax,\tikz@td@Ay,\tikz@td@Az)}%
\edef\tikz@td@lstviscoords{}%
\else
\edef\tikz@td@lsthidcoords{}%
\edef\tikz@td@lstviscoords{(\tikz@td@Ax,\tikz@td@Ay,\tikz@td@Az)}%
\fi
\pgfmathtruncatemacro\pgfutil@tempb{acos(\tikz@td@AdotB)}%
\pgfmathtruncatemacro\pgfutil@tempc{sign(\pgfutil@tempb)}%
\loop
\pgfmathsetmacro{\tmpx}{cos(\pgfutil@tempa)*\tikz@td@Ax-\pgfutil@tempc*sin(\pgfutil@tempa)*\tikz@td@Dx}%
\pgfmathsetmacro{\tmpy}{cos(\pgfutil@tempa)*\tikz@td@Ay-\pgfutil@tempc*sin(\pgfutil@tempa)*\tikz@td@Dy}%
\pgfmathsetmacro{\tmpz}{cos(\pgfutil@tempa)*\tikz@td@Az-\pgfutil@tempc*sin(\pgfutil@tempa)*\tikz@td@Dz}%
\pgfmathtruncatemacro{\pgfutil@tempd}{isfore(\tmpx,\tmpy,\tmpz)}%
\ifnum\pgfutil@tempd=-1\relax
\edef\tikz@td@lsthidcoords{\tikz@td@lsthidcoords\space(\tmpx,\tmpy,\tmpz)}%
\else
\edef\tikz@td@lstviscoords{\tikz@td@lstviscoords\space(\tmpx,\tmpy,\tmpz)}%
\fi
\edef\pgfutil@tempa{\the\numexpr\pgfutil@tempa+1}%
\ifnum\pgfutil@tempa<\the\numexpr\pgfutil@tempc*\pgfutil@tempb\relax
\repeat
\pgfmathtruncatemacro{\pgfutil@tempd}{isfore(\tikz@td@Bx,\tikz@td@By,\tikz@td@Bz)}%
\ifnum\pgfutil@tempd=-1\relax
\edef\tikz@td@lsthidcoords{\tikz@td@lsthidcoords\space(\tikz@td@Bx,\tikz@td@By,\tikz@td@Bz)}%
\else
\edef\tikz@td@lstviscoords{\tikz@td@lstviscoords\space(\tikz@td@Bx,\tikz@td@By,\tikz@td@Bz)}%
\fi
\ifx\tikz@td@lsthidcoords\pgfutil@empty%
\else
\draw[great circle arc/back] plot coordinates {\tikz@td@lsthidcoords};%
\fi
\ifx\tikz@td@lstviscoords\pgfutil@empty%
\else
\draw[great circle arc/fore, #1] plot coordinates {\tikz@td@lstviscoords};%
\fi
}

% ------------------------------------------------------------------------

\newcommand\pgfmathsinandcos[3]{%
  \pgfmathsetmacro#1{sin(#3)}%
  \pgfmathsetmacro#2{cos(#3)}%
}

%
% \LongitudePlane, \LatitudePlane
%
%   defines a plane with the name of the first parameter
%
\newcommand\LongitudePlane[2][current plane]{%
  \pgfmathsinandcos\sinEl\cosEl{\Elevation} % elevation
  \pgfmathsinandcos\sint\cost{#2} % azimuth
  \tikzset{#1/.estyle={cm={\cost,\sint*\sinEl,0,\cosEl,(0,0)}}}
}
\newcommand\LatitudePlane[2][current plane]{%
  \pgfmathsinandcos\sinEl\cosEl{\Elevation} % elevation
  \pgfmathsinandcos\sint\cost{#2} % latitude
  \pgfmathsetmacro\ydelta{\cosEl*\sint}
  \tikzset{#1/.estyle={cm={\cost,0,0,\cost*\sinEl,(0,\ydelta)}}} %
}


% m = mandatory argument, O{x} = optional argument with default value
% examples:
% \DrawLongitudeCircle[45] % longitude circle at 45°
% \DrawLongitudeCircle[45][blue] % longitude circle at 45°, color blue
%
% \DrawLongitudeCircle
%   - depends on \Elevation, \R being defined
\NewDocumentCommand{\DrawLongitudeCircle}{ m O{black}}{%
  \LongitudePlane{#1}
  \tikzset{current plane/.prefix style={scale=\R}}
  \pgfmathsetmacro\angVis{atan(sin(#1)*cos(\Elevation)/sin(\Elevation))} %
  \draw[current plane,thin,#2]  (\angVis:1)     arc (\angVis:\angVis+180:1);
  \draw[current plane,thin,#2,dashed,opacity=0.45] (\angVis-180:1) arc (\angVis-180:\angVis:1);
}

%
% \DrawLatitudeCircle
% 
\NewDocumentCommand{\DrawLatitudeCircle}{ m O{black}}{%
  \LatitudePlane{#1}
  \tikzset{current plane/.prefix style={scale=\R}}
  \pgfmathsetmacro\sinVis{sin(#1)/cos(#1)*sin(\Elevation)/cos(\Elevation)}
  \pgfmathsetmacro\angVis{asin(min(1,max(\sinVis,-1)))}
  \draw[current plane,thin,#2] (\angVis:1) arc (\angVis:-\angVis-180:1);
  \draw[current plane,thin,dashed,#2,opacity=0.45] (180-\angVis:1) arc (180-\angVis:\angVis:1);
}

\makeatother

\definecolor{steelblue}{rgb}{0.273,0.501,0.703} % 70,130,180

% ============================================

\begin{document}

\def\R{2} % in cm
\def\Elevation{20}


\foreach \ViewAngle in {1,3,...,94}{
\begin{tikzpicture}[remember picture,overlay]
%    \node [shift={(5.8 cm, -5.8 cm)}] at (current page.north west)
%     {%
        \begin{tikzpicture}[declare function={R=\R;},bullet/.style={circle,fill,inner sep=2pt}, scale=2.5]
           \path[use as bounding box] (0,0) rectangle (5,5); % Avoid jittering animation
           % draw sphere
           \shade[ball color=lightgray, opacity=0.5] (0,0,0) circle(\R cm);
           %\shade[ball color = white,transform canvas={rotate=-35}] (0,0,0) circle[radius=R];
           
           
           \tdplotsetmaincoords{80}{\ViewAngle} % coord1: 90=head on, 0=top down (but can't go to zero) % 110
           
           \def\qsccorner{35.264389682754654}
           \def\rzero{{\R/sqrt(3)}}
           
           \begin{scope}[tdplot_main_coords]
           
           	%top
              \GreatCircleArc[black, thick] {theta1= \qsccorner, phi1=-45, theta2=\qsccorner,  phi2=45}
              \GreatCircleArc[black, thick] {theta1= \qsccorner, phi1= 45, theta2=\qsccorner,  phi2=135}
              \GreatCircleArc[black, thick] {theta1= \qsccorner, phi1=135, theta2=\qsccorner,  phi2=225}
              \GreatCircleArc[black, thick] {theta1= \qsccorner, phi1=225, theta2=\qsccorner,  phi2=-45}
           
           	%sides
              \GreatCircleArc[black, thick] {theta1= \qsccorner, phi1=45,  theta2=-\qsccorner, phi2=45}
              \GreatCircleArc[black, thick] {theta1= \qsccorner, phi1=135, theta2=-\qsccorner, phi2=135}
              \GreatCircleArc[black, thick] {theta1= \qsccorner, phi1=225, theta2=-\qsccorner, phi2=225}
              \GreatCircleArc[black, thick] {theta1= \qsccorner, phi1=-45, theta2=-\qsccorner, phi2=-45}
           
           	% bottom
              \GreatCircleArc[black, thick] {theta1=-\qsccorner, phi1=-45, theta2=-\qsccorner,  phi2=45}
              \GreatCircleArc[black, thick] {theta1=-\qsccorner, phi1= 45, theta2=-\qsccorner,  phi2=135}
              \GreatCircleArc[black, thick] {theta1=-\qsccorner, phi1=135, theta2=-\qsccorner,  phi2=225}
              \GreatCircleArc[black, thick] {theta1=-\qsccorner, phi1=225, theta2=-\qsccorner,  phi2=-45}
           	
             \begin{scope}[tdplotCsFill/.style={opacity=0.00},
             					 tdplotCsBack/.style={dashed, opacity=0.0}] % set line styles for block
           
                  	\tdplotCsDrawLatCircle[black,thick]{\R}{0}
                  	\tdplotCsDrawLonCircle[black,thick]{\R}{0}
                  	\tdplotCsDrawLonCircle[black,thick]{\R}{90}
               	
               	\def\qlscHalfLat{19.471220634490695}
               	\def\qlscHalfLon{26.565051177077994}
               	\def\qlscHalfLatAtHalfLon{41.810314895778596}
               	
               	\foreach \RA in {45,135,225,315} {
               		% faces 1-4
               		\GreatCircleArc[black, thick] {theta1=\qlscHalfLat, phi1=\RA, theta2=\qlscHalfLat,  phi2=\RA+90} % horiz
               		\GreatCircleArc[black, thick] {theta1=-\qlscHalfLat, phi1=\RA, theta2=-\qlscHalfLat,  phi2=\RA+90} % horiz
               
               		\GreatCircleArc[black, thick] {theta1=\qlscHalfLatAtHalfLon, phi1=\RA-45+\qlscHalfLon, theta2=-\qlscHalfLatAtHalfLon,  phi2=\RA-45+\qlscHalfLon} % vertical
               		\GreatCircleArc[black, thick] {theta1=\qlscHalfLatAtHalfLon, phi1=\RA-45-\qlscHalfLon, theta2=-\qlscHalfLatAtHalfLon,  phi2=\RA-45-\qlscHalfLon} % vertical
               
               		% faces 0 and 5
               		\GreatCircleArc[black, thick] {theta1=\qlscHalfLatAtHalfLon, phi1=\RA-45+\qlscHalfLon, theta2=\qlscHalfLatAtHalfLon,  phi2=\RA-\qlscHalfLon+135}
               		\GreatCircleArc[black, thick] {theta1=-\qlscHalfLatAtHalfLon, phi1=\RA-45+\qlscHalfLon, theta2=-\qlscHalfLatAtHalfLon,  phi2=\RA-\qlscHalfLon+135}
               	}
             \end{scope}
             	
           \end{scope}
        
        \end{tikzpicture}
%    }; % end node
\end{tikzpicture}
\newpage

} % second "}" ends foreach

% end foreach
\end{document}

